\documentclass{article}

% ready for submission
\usepackage{arxiv}

\usepackage[utf8]{inputenc} % allow utf-8 input
\usepackage[T1]{fontenc}    % use 8-bit T1 fonts
\usepackage{hyperref}       % hyperlinks
\usepackage{url}            % simple URL typesetting
\usepackage{booktabs}       % professional-quality tables
\usepackage{amsfonts}       % blackboard math symbols
\usepackage{nicefrac}       % compact symbols for 1/2, etc.
\usepackage{microtype}      % microtypography
\usepackage{amsmath}}

\title{Mimesis as random graph coloring}

\date{April 29, 2019}

% The \author macro works with any number of authors. There are two commands
% used to separate the names and addresses of multiple authors: \And and \AND.
%
% Using \And between authors leaves it to LaTeX to determine where to break the
% lines. Using \AND forces a line break at that point. So, if LaTeX puts 3 of 4
% authors names on the first line, and the last on the second line, try using
% \AND instead of \And before the third author name.

\author{%
  Aidan Rocke\\
  \texttt{aidanrocke@gmail.com} \\
  % examples of more authors
  % \And
  % Coauthor \\
  % Affiliation \\
  % Address \\
  % \texttt{email} \\
  % \AND
  % Coauthor \\
  % Affiliation \\
  % Address \\
  % \texttt{email} \\
  % \And
  % Coauthor \\
  % Affiliation \\
  % Address \\
  % \texttt{email} \\
  % \And
  % Coauthor \\
  % Affiliation \\
  % Address \\
  % \texttt{email} \\
}

\begin{document}

\maketitle

\begin{abstract}
   Inspired by the thought-provoking masterpiece by René Girard, Le Bouc Emissaire, a simple and tractable model for mimetic behaviour occurred to me. When we change our beliefs, we do so not because of their intrinsic value. Our desire to switch from belief $A$ to belief $B$ is proportional to the number of adherents of belief $B$ that we know.
\end{abstract}

\section{Mimesis as a decentralised process}

In this article, I propose that when we change our beliefs, we do so not because of their intrinsic value. Our desire to switch from belief $A$ to belief $B$ is proportional to the number of adherents of belief $B$ that we know. Technically, I modelled the problem of two conflicting beliefs that propagate through a network with $N$ nodes in a decentralised manner. These beliefs are in some sense competing for adherents.

Using vertex notation, two individuals $v_i$ and $v_j$ with identical beliefs are connected with probability $q$, and $1-q$ otherwise. $v_i$ changes its belief
with a probability proportional to the number of nodes connected to $v_i$ that have opposing views.

Two key motivating questions are:

\quad	1. Under what circumstances does a belief get completely wiped out?

\quad	2. Under what circumstances does a belief completely dominate(i.e. wipe out) all other beliefs?

In the scenario where there are only two possible beliefs these two questions are equivalent and I show that on average it's sufficient that
$q > 1-q$ and that initially, one belief has a greater number of adherents than the other.

Nodes carrying the first belief were assigned to the set of red vertices, $R$, and nodes carrying the second belief were assigned to the
set of blue vertices, $B$. After further reflection, I chose $+1$ and $-1$ as labels. The reason being that a change of belief using this representation would be equivalent
to multiplication by $-1$. As a result, the $N$ vertices could be represented by an N-dimensional vector:

\begin{equation}
\vec{v} \in \{-1,1\}^N
\end{equation}

where $N= \lvert v_i \in R \rvert + \lvert v_j \in B \rvert$.


\newpage

\section{A random graph model}

Using this representation, between each pair of vertices we may define a virtual weight matrix $W$:

\begin{equation}
w_{ij} = v_i \cdot v_j
\end{equation}

where $w_{ij}=+1$ implies identical beliefs and we have $w_{ij}=-1$ otherwise.

Now, we note that $W$ may be conveniently decomposed as follows:

\begin{equation}
W= W^+ + W^-
\end{equation}

where $W^-$ denotes potential connections between nodes of different color and $W^+$ denotes potential connections between nodes
of identical colors.

In order to simulate variations in connectivity we may assume that nodes of the same color are connected with probability $\frac{1}{2} < q < 1$ and nodes of different color are connected with probability $1-q$. Given $W$ we may therefore construct the adjacency matrix $A$ by sampling random matrices:

\begin{equation}
M_1, M_2 \sim \mathcal{U}([0,1])^{N \times N}
\end{equation}

\begin{equation}
M^+ = 1_{[0,q)} \circ M_1
\end{equation}

\begin{equation}
M^- = 1_{(1-q,1]} \circ M_2
\end{equation}

where $1_{[0,q)}$ denotes the characteristic function over the set $[0,q)$ and then we compute the Hadamard products:

\begin{equation}
A^+ = M^+ \cdot W^+
\end{equation}

\begin{equation}
A^- = M^- \cdot W^-
\end{equation}

so the adjacency matrix is given by $A = A^+ + A^-$.

\section{Stochastic dynamics for computer simulation}

Now, in order to simulate stochastic dynamics we simply use majority vote:

\begin{equation}
p(v_i^{n+1}=v_i^n) = \frac{\bar{N_i}}{N_i}
\end{equation}

\begin{equation}
p(v_i^{n+1}=-1 \cdot v_i^n) = 1- \frac{\bar{N_i}}{N_i}
\end{equation}

\begin{equation}
\bar{N_i} = \lvert A(i,-) > 0 \rvert -1
\end{equation}

\begin{equation}
N_i = \bar{N_i} + \lvert A(i,-) < 0 \rvert
\end{equation}

where $\lvert A(i,-) > 0 \rvert -1$ denotes the number of connections between $v_i$ and nodes sharing the same belief without
counting a connection to itself.

\newpage 

\section{Analysis}

If we denote the number of red vertices at instant $n$ by $\alpha_n$ and the number of blue vertices by $\beta_n$ we may observe that the expected number of neighbors is given by:

\begin{equation}
\langle N(v_i \in R) \rangle = q \cdot (\alpha_n -1) + (1-q) \cdot \beta_n
\end{equation}

\begin{equation}
\langle N(v_i \in B) \rangle = q \cdot (\beta_n -1) + (1-q) \cdot \alpha_n
\end{equation}

Using the above equations we may define the expected value:

\begin{equation}
\langle \alpha_{n+1} \rangle = \alpha_n \left( \frac{q \cdot (\alpha_n -1)}{q \cdot (\alpha_n -1) + (1-q) \cdot \beta_n} \right) + \beta_n \left(\frac{(1-q) \cdot \alpha_n}{q \cdot (\beta_n -1) + (1-q) \cdot \alpha_n} \right)
\end{equation}

and we may deduce that $\langle \beta_{n+1} \rangle = N - \langle \alpha_{n+1} \rangle$. 

\subsection{$\alpha_n > \beta_n$ implies that $\lim\limits_{n \to \infty} \langle \alpha_n \rangle = N$}

Assuming that $q > 1-q$, a simple calculation shows that:

\begin{equation}
\langle \alpha_{n+1} \rangle - \alpha_n \geq 0 \iff \alpha_n \geq \beta_n
\end{equation}

and since:

\begin{equation}
\langle \alpha_{n+1} \rangle - \alpha_n= 0 \iff \alpha_n = \beta_n
\end{equation}

we may deduce that:

\begin{equation}
\lim\limits_{n \to \infty} \langle \alpha_{n} \rangle = N
\end{equation}

\subsection{Analysis of $\Delta \alpha$}

Using the fact that $\beta_n = N - \alpha_n$ we may derive the following continuous-space variant of $\Delta \alpha_n = \langle \alpha_{n+1} \rangle - \alpha_n$:

\begin{equation}
\Delta \alpha(\alpha,\gamma) = \frac{\alpha \cdot (N-\alpha)}{\gamma \cdot (\alpha -1) + (N-\alpha)} - \frac{\alpha \cdot (N-\alpha)}{\gamma \cdot (N - \alpha -1) + \alpha}
\end{equation}

where $\gamma = \frac{q}{1-q}$.

\section*{References}

\small

[1] Réka Albert & Albert-László Barabási. Statistical mechanics of complex networks. 2002.

[2] René Girard. Le Bouc émissaire. 1986. 

[3] P. Erdös and A. Rényi. On the evolution of random graphs. 1960.

[4] G. Grimmett and C. McDiarmid. On colouring random graphs. 1975.

\end{document}